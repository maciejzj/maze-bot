\documentclass{article}
\usepackage{polski}
\usepackage[utf8]{inputenc}
\usepackage[a4paper,margin=20mm]{geometry}
\usepackage{amsmath}
\usepackage{graphicx}
\graphicspath{ {.//} }
\author{Maciej Ziaja, Bartosz Staszulonek}
\date{\today}
\begin{document}
\title{
  Projekt Systemy Mikroprocesorowe \\
  \large MazeBot \\
    Robot unikający przeszkód z regulacją napędu silników prądu stałego}

\pagenumbering{gobble}
\maketitle
\tableofcontents
\newpage

\section{Wstęp}

\subsection{Abstrakt}
Projekt polegał na budowie mobilnego robota, który unika przeszkód.
Konstrukcja porusza się na dwóch kołach, wykrywa przeszkody za pomocą czujnika ultradźwiękowego zamocowanego na wieży serwomechanizmu.
Platforma robota napędzana jest za pomocą pary silników prądu stałego, których kąt obrotu jest odczytywany przez robota za pomocą czujników szczelinowych.
W celu zwiększenia precyzji działania układu zbudowano układy regulacji prędkości silników.
Wykonano dwa typy układ regulacji, synchronizujący prędkość obrotu obu silników w~celu zachowania kierunku jazdy na wprost oraz kaskadowy układ regulacji skrętu platformy robota.
Układ regulacji skrętu robota przypomina zasadą działania prosty serwomechanizm.
Przedstawiono metodę identyfikacji i~strojenia regulatorów.
Na potrzeby projektu wykonano także schemat układu elektronicznego i~zrealizowano go w~postaci obwodu drukowanego PCB.
Zaprojektowano także podwozie robota i~wydrukowano je w~technologii~3D.

\subsection{Cel i zakres projektu}
\section{Organizacja projektu}
\section{Harmonogram}
\section{Budżet}

\section{Budowa prototypu, analiza problemów}

\section{Projekt układu elektronicznego}
\subsection{Schemat układu elektronicznego}
\subsection{Projekt układu drukowanego}


\section{Projekt podwozia robota}

\section{Synteza układów regulacji}
\subsection{Regulator synchronizacji prędkości silników}
\subsection{Kaskadowy regulator skrętu platformy}
\subsection{Identyfikacja obiektu regulacji}
\subsection{Strojenie regulatorów}

\section{Implementacja programistyczna}

\section{Wykorzystane technologie}

\section{Perspektywy rozwoju, podsumowanie}


\pagenumbering{arabic}
\end{document}